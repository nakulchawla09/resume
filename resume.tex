% The font could be set to Windows-specific Calibri by using the 'calibri' option
\documentclass[]{mcdowellcv}

% For mathematical symbols
\usepackage{xcolor}
\usepackage{hyperref}
% \usepackage[colorlinks = true,
%             linkcolor = blue,
%             urlcolor  = blue,
%             citecolor = blue,
%             anchorcolor = blue]{hyperref}
\definecolor{myblue}{HTML}{00008B}
\hypersetup{
  linkcolor  = myblue,
  citecolor  = myblue,
  urlcolor   = myblue,
  colorlinks = true,
}


% Set applicant's personal data for header
\name{Nakul Chawla}
\address{\href{https://nakul.net}{https://nakul.net} \linebreak
\href{https://linkedin.com/in/thenakulchawla}{https://linkedin.com/in/thenakulchawla}}
\contacts{Cell: (480) 930-6967  \linebreak Email: \href{mailto:nchawla3@asu.edu}{nchawla3@asu.edu}}

\begin{document}

% Print the header
    \makeheader
    % Print the content
    \begin{cvsection}{Work Experience}
        \begin{cvsubsection}{Software Eng. Intern}{\href{https://www.dash.org/}{Dash Core Group, Inc}}{05/2018 --
            08/2018}
            \begin{cvsubsectionitem}
            \item Implemented Graphene block propagation, that decreases block payload by encoding the already
                broadcasted transactions into a bloom filter and an Inverted bloom lookup table (IBLT), keeping the False
                positive rate(FPR) adjustable by the sender .
            \item In order to reduce the overall bandwidth used by a block to propagate, both bloom filter that is
                associated with an FPR, and IBLT that uses 17 bytes per transaction, are used.
            \item Developed Express validation for blocks, and prevents DoS attacks against Expedited blocks.
            \item Acquired deep understanding of decentralized peer to peer networks. 
            % \item Technologies: \emph{C++, Python}
            \end{cvsubsectionitem}
        \end{cvsection}
        \begin{cvsubsection}{Research Assistant}{\href{https://blockchain.asu.edu}{Blockchain Research Lab,
            ASU}}{08/2017 -- present}
            \begin{cvsubsectionitem}
            \item Developed a simulator for Dash network using Network Simulator 3 to choose a propagation technology between
                xthin and compact by analyzing orphan rates for higher block sizes (upto 10MB). -- \href{https://github.com/thenakulchawla/dash-simulator}{dash-simulator}
            \item The research addresses scalability issues with Dash network by showing that xthin(10MB block
                size) propagation increased throughput to 120 tx/sec (2.5x faster than compact(4MB block size)) at profitable mining rates.
            % \item Technologies: \emph{C++, Network Simulator 3.25, Python waf}
            \end{cvsubsectionitem}
        \end{cvsubsection}

        \begin{cvsubsection}{Developer BI}{Target Corporation India}{07/2013 -- 07/2016}
            \begin{cvsubsectionitem}
            \item Designed and developed full-stack (Rest APIs, MVVM) application for Real Estate tax Analytics team.
                The application provided a platform for users to automatically file taxes in time, which was previously
                done manually using spreadsheet.
            \item Achieved faster data processing by optimizing advanced SQL queries.
            \item Responsible for designing and database  warehousing for a type 2 financial vendor income database to
                store historical information.
            % \item Technologies: \emph{C\#, AngularJs, SQL-Server, Teradata, Hadoop-Spark}
            \end{cvsubsectionitem}
        \end{cvsubsection}
    \end{cvsection}

    \begin{cvsection}{Education}
        \begin{cvsubsection}{M.S. Computer Science}{Arizona State University, Tempe, AZ}{08/2016 -- 12/2018}
            \textbf{GPA: } 3.30 \\
         \textbf{Thesis:} Blockchains Proof of Work Sharding - Distributed storage for blockchains to reduce overall
            memory footprint of a full node in a peer to peer network. \\
            \textbf{Relevant Coursework:} Advanced Memory Systems, Advanced Database Systems, Distributed Database Systems, Distributed
            and Multiprocessor Operating Systems, Algorithms and Data structures, Principles of Programming Language
        \end{cvsubsection}
        \begin{cvsubsection}{B.E. Information Science}{M.S Ramaiah Institute of Technology}{08/2009 -- 05/2013}
            \textbf{Relevant Coursework:} Computer Networks, Operating Systems
        \end{cvsubsection}
    \end{cvsection}

    \begin{cvsection}{Technical Skills}
        \begin{cvsubsection}{}{}{}
            \begin{cvsubsectionitem}
                \item Python, Linux, C++, SQL, C, REST, AutoTools, JavaScript, Docker, C\#, AngularJs, NodeJs, Git,
                    .Net, IIS, docker.
            \end{cvsubsectionitem}
        \end{cvsubsection}
    \end{cvsection}

    \begin{cvsection}{Projects}
        \begin{cvsubsection}{}{}{}
            \begin{cvsubsectionitem}
            \item \itextbf{In-Memory Spatial Database Index}
                \begin{itemize}
                    \item Developed PR-Quadtrees to index the space for the in-memory database.
                    \item Placed the pointers of a child quadnode contiguously in a vector collection instead of having
                        one pointer to node like the classic implementation. This made the QuadNode have a smaller
                        memory footprint in the cache and improved data locality.
                    \item Improving data locality resulted in descendants present in the cache increasing the cache
                        hit-cache miss ratio. 
                    \item For MXCIF Quadtree, the leaf capacity was made configurable, in order to reduce the fan out on
                        insertion of new data. 
                    \item  The approach overall boosted the space usage by approx. 40\%.
                    \item Technologies: \emph{C++}
                \end{itemize}
            \end{cvsubsectionitem}
        \end{cvsubsection}
    \end{cvsection}

    \begin{cvsection}{Publications}
        \begin{cvsubsection}{}{}{}
            \begin{cvsubsectionitem}
            \item Whitepaper: Block Propagation Applied to Nakamoto Consensus.
                \href{https://blockchain.asu.edu/block-propagation-applied-to-nakamoto-networks} {Block Propagation
                Applied to Nakamoto Consensus}
            \end{cvsubsectionitem}
        \end{cvsubsection}
    \end{cvsection}

    % \begin{cvsection}{Additional Experience and Awards}
    %     \begin{cvsubsection}{}{}{}
    %         \begin{cvsubsectionitem}
    %         \item Featured in Thrive magazine at ASU for conducting research in blockchains. -- \href{https://asunow.asu.edu/20180327-solutions-asu-engineers-data-defenders}{Data defender}
    %         \end{cvsubsectionitem}
    %     \end{cvsubsection}
    % \end{cvsection}

\end{document}
