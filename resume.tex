% The font could be set to Windows-specific Calibri by using the 'calibri' option
\documentclass[]{mcdowellcv}

% For mathematical symbols
\usepackage{xcolor}
\usepackage{hyperref}
% \usepackage[colorlinks = true,
%             linkcolor = blue,
%             urlcolor  = blue,
%             citecolor = blue,
%             anchorcolor = blue]{hyperref}
\definecolor{myblue}{HTML}{00008B}
\hypersetup{
  linkcolor  = myblue,
  citecolor  = myblue,
  urlcolor   = myblue,
  colorlinks = true,
}


% Set applicant's personal data for header
\name{Nakul Chawla}
\address{\href{https://nakul.net}{https://nakul.net} \linebreak \href{https://github.com/thenakulchawla}{https://github.com/thenakulchawla}}
\contacts{Cell: (480) 930-6967  \linebreak Email: \href{mailto:nakulchawla09@gmail.com}{nakulchawla09@gmail.com}}

\begin{document}

% Print the header
    \makeheader

    % Print the content
    \begin{cvsection}{Employment}
        \begin{cvsubsection}{Research Assistant}{\href{https://blockchain.asu.edu}{Blockchain Research Lab, ASU}}{08/2017 -- 05/2018}
            \begin{cvsubsectionitem}
            \item Developed Dash-Simulator using Network Simulator 3 to help scale Dash to the next block propagation technique and block size with profitable orphan rates -- \href{https://github.com/thenakulchawla/dash-simulator}{dash-simulator}
            \item Developed Compact and Xthin block propagation for the dash-simulator.
            \item Simulated traditional, Compact and xthin for 750 (for better confidence interval) blocks each and
                analyzed orphan rates.-- \href{https://github.com/thenakulchawla/dash-results}{Results}
            \item Technologies: \emph{C++, Network Simulator 3.25}
            \end{cvsubsectionitem}
        \end{cvsubsection}
        \begin{cvsubsection}{Software Eng. Intern}{\href{https://www.dash.org/}{Dash Core Group, Inc}}{05/2018 -- present}
            \begin{cvsubsectionitem}
            \item Implemented Graphene block propagation, that utilizes set reconciliation instead of sending all
                transaction IDs that the receiver already has when the transactions are broadcast.
            \item Graphene comprises of bloom filter of all transaction IDs in a bloom filter and an Inverted bloom
                lookup table created using the bloom filter. The False positive rate of bloom filter is set by the
                Sender.
            \item The FPR of the bloom filter can be set high in lue that IBLT recovers the information efficiently.
            \item The IBLT takes 17 bytes per transaction for encoding, therefore, only IBLT is not used for encoding
                the transactions.
            \item Developed Express validation for blocks, and prevents DoS attacks against Expedited blocks.
            \item Technologies: \emph{C++, Python}
            \end{cvsubsectionitem}
        \end{cvsection}

        \begin{cvsubsection}{Developer BI}{Target Corporation India}{07/2013 -- 07/2016}
            \begin{cvsubsectionitem}
            \item Developed full-stack (Rest APIs, MVVM) application for Real Estate tax Analytics team.
            \item Implemented Database Factory, Unit of Work and Disposable patterns.
            \item Optimized SQL queries and designed/warehoused the database on SQL Server.
            \item Developed REST APIs for the Real Estate Tax Application.
            \item Technologies: \emph{C\#, AngularJs, SQL-Server, Teradata, Hadoop-Spark}
            \end{cvsubsectionitem}
        \end{cvsubsection}
    \end{cvsection}

    \begin{cvsection}{Education}
        \begin{cvsubsection}{M.S. Computer Science}{Arizona State University, Tempe, AZ}{08/2016 -- 05/2018}
         \textbf{Thesis:} Blockchains Proof of Work Sharding - Shared/dispersed storage for blockchains.  \\
            \textbf{Relevant Coursework:} Advanced Memory Systems, Advanced Database Systems, Distributed Database Systems, Distributed
            and Multiprocessor Operating Systems, Algorithms and Data structures, Principles of Programming Language
        \end{cvsubsection}
        \begin{cvsubsection}{B.E. Information Tech.}{M.S Ramaiah Institute of Technology}{08/2009 -- 05/2013}
        \end{cvsubsection}
    \end{cvsection}

    \begin{cvsection}{Technical Skills}
        \begin{cvsubsection}{}{}{}
            \begin{cvsubsectionitem}
                \item Python, Linux, C++, SQL, C, REST, AutoTools, JavaScript, Docker, C\#, AngularJs, NodeJs, Git,
                    .Net, IIS.
            \end{cvsubsectionitem}
        \end{cvsubsection}
    \end{cvsection}

    \begin{cvsection}{Projects}
        \begin{cvsubsection}{}{}{}
            \begin{cvsubsectionitem}
            \item \itextbf{In-Memory Spatial Database Index}
                \begin{itemize}
                    \item Developed PR-Quadtrees to index the space for the in-memory database.
                    \item Placed the pointers of a child quadnode contiguously in a vector collection instead of having
                        one pointer to node like the classic implementation. This made the QuadNode have a smaller
                        memory footprint in the cache and improved data locality.
                    \item Improving data locality resulted in descendants present in the cache increasing the cache
                        hit-cache miss ratio. 
                    \item For MXCIF Quadtree, the leaf capacity was made configurable, in order to reduce the fan out on
                        insertion of new data. 
                    \item  The approach overall boosted the space usage by approx. 40\%.
                    \item Technologies: \emph{C++}
                \end{itemize}
            \end{cvsubsectionitem}
        \end{cvsubsection}
    \end{cvsection}

    \begin{cvsection}{Publications}
        \begin{cvsubsection}{}{}{}
            \begin{cvsubsectionitem}
            \item Whitepaper: Block Propagation Applied to Nakamoto Consensus.
                \href{https://blockchain.asu.edu/block-propagation-applied-to-nakamoto-networks} {Block Propagation
                Applied to Nakamoto Consensus}
            \end{cvsubsectionitem}
        \end{cvsubsection}
    \end{cvsection}

    \begin{cvsection}{Additional Experience and Awards}
        \begin{cvsubsection}{}{}{}
            \begin{cvsubsectionitem}
            \item Featured in Thrive magazine at ASU for conducting research in blockchains. -- \href{https://asunow.asu.edu/20180327-solutions-asu-engineers-data-defenders}{Data defender}
            \end{cvsubsectionitem}
        \end{cvsubsection}
    \end{cvsection}

\end{document}
